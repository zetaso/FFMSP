t_ejec <- 0 (tiempo de ejecucion actual)
solutions <- define una lista de strings soluciones
padres[4] <- lista temporal de 4 soluciones
torneoA[2] <- lista temporal de 2 soluciones
torneoB[2] <- lista temporal de 2 soluciones

for i <- 0 to population_size do
    if method == 0 then
        solutions[i] <- greedy()
    else then
        solutions[i] <- pgreedy()
    i <- i+1
    
while t_ejec < t then
    padres <- elegimos 4 soluciones distintas aleatoriamente
    torneoA <- elegimos las primeras 2 soluciones de padres
    torneoB <- elegimos las ultimas 2 soluciones de padres

    padreA <- buscamos la mejor solucion de torneoA
    padreB <- buscamos la mejor solucion de torneoB
    hijoA <- solucion auxiliar
    hijoB <- solucion auxiliar

    peores[2] <- lista de soluciones auxiliares

    pivot <- valor aleatorio entre 0 y columns-1
	for i <- 0 to columns do
		if i < pivot then
            hijoA[i] <- padreA[i]
			hijoB[i] <- padreB[i]
		else then
            hijoA[i] <- padreB[i]
			hijoB[i] <- padreA[i]
        i <- i+1

    rand_pos <- valor aleatorio entre 0 y columns-1
    Con una determinada probabilidad se reemplaza
    el caracter en hijoA[rand_pos] por A, T, G o C.

    rand_pos <- valor aleatorio entre 0 y columns-1
    Con una determinada probabilidad se reemplaza
    el caracter en hijoB[rand_pos] por A, T, G o C.

    peores <- buscamos las 2 peores soluciones
    if score(hijoA) > score(peores[0]) then
        peores[0] <- hijoA

    if score(hijoB) > score(peores[1]) then
        peores[1] <- hijoB

    t_ejec <- t_ejec + delta_t